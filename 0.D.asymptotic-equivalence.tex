In genetic association studies, researchers have the freedom to choose their favorite statistical procedure to perform hypothesis tests with the data collected.
In this section, we answer the natural question that arises: does the choice of statistical test influence the power of scientific discovery?

It turns out, perhaps unsurprisingly, that most common association tests have asymptotically equivalent power. 
We shall quantify this shared power limit, and provide practical formulas for power calculations.

\subsection{A model-invariant parametrization}

In association studies for qualitative traits, counts of subjects in each phenotype-genetic variant combination are tabulated in the form of a contingency table.
For a 2-allele-variant-by-2-phenotype definition, we have the following table of counts.

\begin{center}
    \begin{tabular}{cccc}
    \hline
    & \multicolumn{2}{c}{Genotype} & \\
    \cline{2-3}
    \# Observations & Variant 1 & Variant 2 & Total by phenotype \\
    \hline
    Cases & $O_{11}$ & $O_{12}$ & $n_1$ \\
    Controls & $O_{21}$ & $O_{22}$ & $n_2$ \\
    \hline
    \end{tabular}
\end{center}

Statistics are then calculated based on the counts, to test for associations between the genotypes and phenotypes, at levels adjusted for multiplicity.
Performance of a test is measured in terms of power, i.e., probability of correct rejection under an alternative hypothesis.

As we have seen in Section \ref{sec:disease-models}, power analysis typically starts by assuming an alternative distribution, typically (though not necessarily) described by a disease model.
Power of a test is approximated either based on large sample asymptotics, or by simulating the empirical distribution of the statistic under the alternative.

Recall the 2-by-2 multinomial distributions with probability matrix $\mu = (\mu_{ij})_{2\times2}$,
\begin{center}
    \begin{tabular}{cccc}
    \hline
    & \multicolumn{2}{c}{Allele type} \\
    \cline{2-3}
    Prob. in study & Risk allele & Non-risk allele & Total by phenotype \\
    \hline
    Cases & $\mu_{11}$ & $\mu_{12}$ & $\phi = \mu_{11} + \mu_{12}$ \\
    Controls & $\mu_{21}$ & $\mu_{22}$ & $1-\phi = \mu_{21} + \mu_{22}$ \\
    \hline
    \end{tabular}
\end{center}

We may assume -- by relabelling, and hence without loss of generality -- that allele Variant 1 is positively associated with the Cases, and referred to as the risk allele/variant. 

The multinomial probability matrix $\mu$ can be fully parametrized by the parameter triple:

\begin{itemize}
    \item Fraction of Cases $\phi$, i.e., marginal distribution of phenotypes.
    \item Conditional distribution of risk variant among Controls, i.e., risk allele frequency (RAF) in the Control group 
    \begin{equation} \label{eq:risk-allele-frequency}
    f := \mu_{21}/(1-\phi).
    \end{equation}
    \item Odds ratio (OR) of the allele Variant 1 to Variant 2
    \begin{equation} \label{eq:odds-ratio}
    \text{R} := \frac{\mu_{11}}{\mu_{21}}\Big/\frac{\mu_{12}}{\mu_{22}}
    = \frac{\mu_{11}\mu_{22}}{\mu_{12}\mu_{21}}.
    \end{equation}
\end{itemize}

An alternative hypothesis, e.g., a disease model, determines the canonical parameters $(f, R)$ implicitly,
%The fraction of Cases $\phi$ is part of the study design.
and therefore fully determines statistical power for a specific test at given sample sizes.
Alternatively, power can also be calculated by directly prescribing the canonical parameters and the sample sizes.

It is worth pointing out that while disease models play no role beyond specifying the alternative, they do sometimes inform our choice of a test statistic, hence influencing statistical power in higher order contingency tables.
These tests include, e.g., the Cochran-Armitage test, and variations thereof;
see \cite{Gonzalez08, Li08} for further examples where tests are tailored to disease models. 

We make the important distinction between RAF \emph{in the Control group} ($f$), versus RAF \emph{in the study} ($\mu_{11}+\mu_{21}$), and RAF \emph{in the general population} ($p$).
In the following sections, unless otherwise stated, RAF will refer to the risk allele frequency in the Control group, consistent with the reporting standards of the NHGRI-EBI Catalog \citep{MacArthur16}.

\subsection{Conditional vs unconditional tests}

Readers familiar with the underlying assumptions of association tests in contingency tables may have noticed that we have described a multinomial distribution of the cell counts.
That is, we have only conditioned on the total number of observations in the study.
This is indeed the assumption behind tests such as (the original, unconditional version of) the likelihood ratio test, and the Person chi-square test.

It is not, however, the assumption behind some other tests.
For example, analysis of t-tests typically assumes the observed number in each arm of the study are given.
That is, we would condition on the phenotype marginals when comparing the proportions of genetic variants among the Cases and Controls.
It is perhaps an assumption most close to reality, where the number of samples collected in each arm of the study are pre-determined.
In this case, we have two binomial observations, Binom$(n_1, p_1)$ and Binom$(n_2, p_2)$, instead of a multinomial observation.
\begin{center}
    \begin{tabular}{cccc}
    \hline
    & \multicolumn{2}{c}{Allele type} \\
    \cline{2-3}
    Cond. Prob. & Variant 1 & Variant 2 & Counts by phenotype \\
    \hline
    Cases & $p_{1}$ & $1-p_{1}$ & $n_1$ \\
    Controls & $p_{2}$ & $1-p_{2}$ & $n_2$ \\
    \hline
    \end{tabular}
\end{center}
RAF and OR can be similarly defined,
\begin{itemize}
    \item Marginal distribution of Cases, fixed at $\phi := n_1/(n_1+n_2)$,
    \item Risk allele frequency (RAF) in the Control group is the synonymous with the conditional distribution of risk variant among Controls, $$f := p_2,$$
    \item Odds ratio (OR)
    $$\text{R} := \frac{p_{1}\phi}{p_{2}(1-\phi)}\Big/\frac{(1-p_{1})\phi}{(1-p_{2})(1-\phi)}
    = \frac{p_{1}(1-p_{2})}{(1-p_{1})p_{2}}.
    $$
\end{itemize}
Alternative hypotheses may be formed as in the multinomial case with parameters $\phi$, $f$ and $R$, along with the total samples size.

Finally, we mention the assumptions behind the Fisher's exact test.
The Fisher exact test conditions on the number of observations of both the phenotype variants and genetic variants, leading to a hypergeometric distribution of the first cell count $O_{11}$ given the marginals $n_1$, $n_2$, and $O_{11}+O_{21}$, under the null hypothesis.
We found no easy parametrizations of alternative hypotheses under this framework.
Indeed, existing power calculations for Fisher's exact test resort to simulations under the two-binomial assumptions \citep{Smyth17}.

We refer interested readers to the recent work by \citet{Ripamonti17} and \citet{Choi17} which elucidate the controversies regarding the choices of conditioning when performing statistical inferences on 2-by-2 contingency tables.
We do not attempt to resolve the controversies in this work.
Our goal is to state clearly the assumptions behind the tests, and show the asymptotic equivalence in terms of power, under their respective assumptions.

\subsection{A test-independent power analysis}
\label{subsec:power-calculation}

We now present the main result allowing a unified power analysis, applicable for a wide range of common association tests in 2-by-2 tables.
% This is formalized in Theorem \ref{thm:1} below, where we show a number of tests enjoy the same power asymptotically.
% This unifies, and simplifies power calculations.

If we consider a fixed parameter values of $(f, R)$ under the alternative, no matter how close to the null, the probability of rejection of the null hypothesis by any reasonable test should approach one as sample size increases ($n\to\infty$).
On the other hand, the probability of rejection is less than one in finite samples, making this type of asymptotics useless for approximation.

Therefore, in order to find finer approximations of power, we study alternatives close to the null.
In particular, we take a sequence of alternatives approaching a limit point in the null space, in the hope that limiting rejection probability is between 0 and 1.
It turns out -- see, e.g., \citet{Ferguson17} Chapter 10, and \citet{Lehmann04} Chapter 5 -- that the appropriate rate at which the alternatives should shrink towards the limit point is $1/\sqrt{n}$. % and the appropriate limit point is the maximum likelihood estimate (MLE) of the probabilities under the null hypothesis (which happens to coincide with the minimum chi-square estimates).

Under the multinomial assumption, let $\mu^{(0)}$ be the probability matrix of the (independent) 2-by-2 multinomial distribution, with marginals $(\theta, 1-\theta)$ for the genetic variants, and marginals $(\phi, 1-\phi)$ for the phenotypes.
We require that $\theta\in(0,1)$ and $\phi\in(0,1)$ be bounded away from 0 and 1.
Let $\mu = \mu^{(n)}$ be the sequence of alternatives such that 
\begin{equation} \label{eq:alternative-multinomial}
    \sqrt{n}(\mu^{(n)} - \mu^{(0)}) \rightarrow \delta \begin{pmatrix} 1 & -1 \\ -1 & 1 \end{pmatrix},
\end{equation}
where $\delta$ is a positive constant.

Equivalently, under the two-binomial assumption, let $p_1^{(0)} = p_2^{(0)} = \theta$ be the null hypothesis, with fixed marginals $(\phi, 1-\phi)$ for the phenotypes.
Let $(p_1, p_2) = p^{(n)}$ be the sequence of alternatives such that 
\begin{equation} \label{eq:alternative-two-bionomial}
    \sqrt{n}\phi (p_1-\theta) \rightarrow \delta
    \quad \text{and} \quad 
    \sqrt{n}(1-\phi) (p_2-\theta) \rightarrow -\delta,
\end{equation}
where $\delta$ is a positive constant.
It is easy to see that with the same $\delta$ the two sequences of alternatives have the same RAF and OR, and therefore have the same expected number of observations in each cell.

\begin{theorem} \label{thm:1}
In 2-by-2 contingency tables, under the assumption that the counts in the contingency table follow the multinomial distributions.
%\vspace{-5pt}
\begin{itemize}
    \item The likelihood ratio test for independence,
    \item the likelihood ratio test for zero slope in logistic regressions,
    \item Person's chi-squared test for Independence,
\end{itemize}
%\vspace{-5pt}
and under the assumption that counts in the contingency table follow the two binomial distributions, 
%\vspace{-5pt}
\begin{itemize}
    \item the two-sided Welch's t-test for equal proportions 
\end{itemize} 
%\vspace{-5pt}
have the same asymptotic power curves.
Specifically, for the sequence of alternatives defined in \eqref{eq:alternative-multinomial} and \eqref{eq:alternative-two-bionomial}, all of the listed tests, at level $\alpha$, have statistical powers converging to 
\begin{equation} \label{eq:power-approx}
    \mathbb P[\chi^2(\lambda) \ge q_{\alpha}],
\end{equation}
where $q_{\alpha}$ is the upper $\alpha$ quantile of the central chi-square distribution, and $\chi^2(\lambda)$ is a non-central chi-square distribution with non-centrality parameter
\begin{equation} \label{eq:non-centrality}
    \lambda = \delta^2/{(\theta(1-\theta)\phi(1-\phi))}.
\end{equation}
\end{theorem}

The proof of Theorem \ref{thm:1} is detailed in Section \ref{sec:proof} below.

Theorem \ref{thm:1} is the central result that paves the way for a unified power analysis.
It allows us to chart findings from different studies employing the applicable tests in the same diagram, with the same power limits.
In particular, for large samples, tests for zero slopes in logistic regressions should report approximately the same set of loci as Welch's t-tests for equal proportions on the same dataset, after the same family-wise error rate adjustments.
The estimated odds ratios (in the case of logistic regression, estimate slopes exponentiated) and RAF's, when charted on the OR-RAF diagram, should also follow the same power limits.

To use this result for power calculations, we start with an alternative hypothesis, defined by the canonical parameters $(f, R)$, and sample sizes $(\phi, n)$.
\begin{center}
    \begin{tabular}{ccc}
    \hline
    & \multicolumn{2}{c}{Allele type} \\
    \cline{2-3}
    Probabilities & Variant 1 & Variant 2 \\
    \hline
    Cases & ${fR\phi}/{(fR+1-f)}$ & ${(1-f)\phi}/{(fR+1-f)}$ \\
    Controls & $f(1-\phi)$ & $(1-f)(1-\phi)$ \\
    \hline
    \end{tabular}
\end{center}
Elementary algebra yields $\theta = {fR\phi}/{(fR+1-f)} + f(1-\phi)$, and $\delta = \sqrt{n}(\theta - f)(1-\phi)$.
If tests are based on allele type counts, accounting for the fact that each genetic location has a pair of alleles, the effective sample sizes should be doubled, and the appropriate non-centrality parameter becomes  $\delta = \sqrt{2n}(\theta - f)(1-\phi)$.
Power may then be approximated using the formula in \eqref{eq:power-approx}.
% We conjecture that all aforementioned tests are asymptotically equivalent, in both the multinomial and two-binomial framework.


