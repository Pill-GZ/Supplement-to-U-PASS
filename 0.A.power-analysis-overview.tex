In power analysis of genetic association studies, disease models are used to specify the distribution of observations under the alternative hypothesis.
In this document, we discuss the challenges in disease model specifications, and discuss its relationship with an alternative approach that accomplishes the same goal through a set of "cannonical parameters".

We shall first briefly recall the main steps of a typical power analysis for genetic association studies.

\subsection{Disease model specifications}

A typical power analysis for genetic association studies begin by specifiying an alternative hypothesis through a disease model (dominant, recessive, multiplicative, additive, etc.), which assumes:

\begin{itemize}
    \item The genotype relative risks (GRR).
    \item Risk allele frequency in the general population ($p$).
    \item Disease prevalence in the general population (Prev).
\end{itemize}

The disease model and parameters determine the joint distribution of the genotypes and phenotyes {\it in the population}, shown in the following table.

\begin{center}
    \begin{tabular}{cccc}
    \hline
    & \multicolumn{3}{c}{Risk allele copies} \\
    \cline{2-4}
    Population Prob. & 0 copies & 1 copy & 2 copies \\
    \hline
    Cases & $\pi_{10}$ & $\pi_{11}$ & $\pi_{12}$ \\
    Controls & $\pi_{20}$ & $\pi_{21}$ & $\pi_{22}$ \\
    \hline
    \end{tabular}
\end{center}

In the disease models, the conditional probabilities of having the disease, given the risk allele copy numbers, satisfy the following relations,

\begin{equation} \label{eq:GRR}
    \frac{\pi_{10}}{\pi_{10} + \pi_{20}} : \frac{\pi_{11}}{\pi_{11} + \pi_{21}} : \frac{\pi_{12}}{\pi_{12} + \pi_{22}}
    = \begin{cases}
    1 : \text{GRR} : \text{GRR}^2, &\text{Multiplicative} \\
    1 : \text{GRR} : 2\times\text{GRR}-1, & \text{Additive} \\
    1 : \text{GRR} : \text{GRR}, & \text{Dominant} \\
    1 : 1 : \text{GRR}, & \text{Recessive}
\end{cases}
\end{equation}
where GRR is strictly greater than 1 under the alternative, and equal to 1 under the null hypothesis.

The disease prevalence determines the sum of the probabilities of cases in the population,
\begin{equation} \label{eq:prev}
    \pi_{10} + \pi_{11} + \pi_{12} = \text{Prev}.
\end{equation}

The risk allele frequency in the general population, $p$, assuming Hardy-Weinberg equilibrium, satisfies 
\begin{equation} \label{eq:RAF-pop}
    \pi_{10} + \pi_{20} = (1-p)^2, \quad \pi_{11} + \pi_{21} = 2p(1-p), \quad \pi_{12} + \pi_{22} = p^2.
\end{equation}

The population probabilities are determined by the disease model and its parameters (GRR, Prev, and $p$).
The six unknowns  $(\pi_{10},\ldots,\pi_{22})$ and are solved for using the six equations above: two from Relation \eqref{eq:GRR}, one from \eqref{eq:prev}, and three from \eqref{eq:RAF-pop}.


\subsection{Sampling adjustments}

Next, the probabilities of observing each genotype-phenotype combination are adjusted according the number of cases and controls recruited *in the studies*, where the sample sizes are specified with

\begin{enumerate}
    \item The number of cases ($n_1$) and controls ($n_2$), or equivalently, the fraction of cases ($\phi$) and total number of subjects ($n$).
\end{enumerate}


\begin{center}
    \begin{tabular}{cccc}
    \hline
    & \multicolumn{3}{c}{Risk allele copies} \\
    \cline{2-4}
    Prob. in study & 0 copies & 1 copy & 2 copies \\
    \hline
    Cases & $\pi_{10}\frac{\phi}{\text{Prev}}$ & $\pi_{11}\frac{\phi}{\text{Prev}}$ & $\pi_{12}\frac{\phi}{\text{Prev}}$ \\
    Controls & $\pi_{20}\frac{1-\phi}{1-\text{Prev}}$ & $\pi_{21}\frac{1-\phi}{1-\text{Prev}}$ & $\pi_{22}\frac{1-\phi}{1-\text{Prev}}$ \\
    \hline
    \end{tabular}
\end{center}

As an example, if $\phi > \text{Prev}$, the probabilities are adjusted to account for over-sampling of cases.

The relative frequencies of allele type-phenotype combinations {\it in the study} are then calculated as follows.

\begin{center}
    \begin{tabular}{ccc}
    \hline
    & \multicolumn{2}{c}{Allele variant} \\
    \cline{2-3}
    Prob. in study & Risk allele & Non-risk allele \\
    \hline
    Cases & $\phi\left(\frac{\pi_{12}}{\text{Prev}}+\frac{\pi_{11}}{2\times\text{Prev}}\right)$ & $\phi\left(\frac{\pi_{11}}{2\times\text{Prev}}+\frac{\pi_{10}}{\text{Prev}}\right)$ \\
    Controls & $(1-\phi)\left(\frac{\pi_{22}}{1-\text{Prev}}+\frac{\pi_{21}}{2(1-\text{Prev})}\right)$ & $(1-\phi)\left(\frac{\pi_{21}}{2(1-\text{Prev})}+\frac{\pi_{20}}{1-\text{Prev}}\right)$ \\
    \hline
    \end{tabular}
\end{center}

To simplify notation, we denote the relative frequencies of allele type-phenotype combinations with $\mu = (\mu_{11}, \mu_{12}, \mu_{21}, \mu_{22})$.


\begin{center}
    \begin{tabular}{cccc}
    \hline
    & \multicolumn{2}{c}{Allele variant} & \\
    \cline{2-3}
    Prob. in study & Risk allele & Non-risk allele & Total by phenotype \\
    \hline
    Cases & $\mu_{11}$ & $\mu_{12}$ & $\phi = \mu_{11} + \mu_{12}$ \\
    Controls & $\mu_{21}$ & $\mu_{22}$ & $1-\phi = \mu_{21} + \mu_{22}$ \\
    \hline
    \end{tabular}
\end{center}

\subsection{Power calculations of association tests}

Finally, the power of an statistical test is calculated as the probability of (a correct) rejection, assuming that the data (i.e., tabulated counts of the allele type-phenotype combinations) follow a multinomial distribution with probability matrix $\mu$ and sample size $2n$, since each individual has a pair of alleles.

% \footnote{Caveat: The final step in the power calculations does not apply to association tests performed directly on the 2-by-3 contingency tables of phenotype-genotype combinations (e.g., the Cochran-Armitage test).Therefore, the GAS calculator should not be used if these tests are to be applied.}.

Some common association tests include the likelihood ratio test, Pearson's chi-square test, tests of zero slope coefficient in logistic regressions, as well as t-tests for equal proportions.
In principal, power analysis has to be tailored to the association test used.
Fortunately, many of these tests are asymptotically equivalent in terms of power, and results of the power analysis applies to all equivalent tests; see Section \ref{sec:asymptotic-equivalence}.

These steps form the basis of the calculations implemented in the most existing tools, including the GAS calculator \citep{Johnson17}.
Although not explicitly stated, the GAS calculator assumes the test of association to be the Welch t-test.

