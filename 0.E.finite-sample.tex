
The result of our power calculations above are only accurate to the extend that the asymptotic approximations are applicable.
In practice, of course, we have only finite samples, and the asymptotic approximations no longer hold when cell counts are low.
While existing tools have completely ignored this issue, we offer here a simple correction in finite samples by resorting to exact tests.

Specifically, we calculate the minimum number of observations of the genetic variants needed for Fisher's exact test to be correctly calibrated, referred to as the {\bf minimum calibration numbers}.
As we shall see in simulations in Section \ref{sec:numerical}, they provide a useful lower bound on the variant counts necessary for any asymptotic approximations to apply.

For a contingency table with marginal phenotype counts $(n_1, n_2)$, and marginal genetic variant counts $(m_1, m_2)$, we calculate the p-values of the most extreme observations according to Fisher's exact test.
For rare risk alleles, this corresponds to the following table.
\begin{center}
    \begin{tabular}{cccc}
    \hline
    & \multicolumn{2}{c}{Allele type} \\
    \cline{2-3}
    \# Observations & Variant 1 & Variant 2 & Counts by phenotype \\
    \hline
    Cases & $m_1$ & $n_1-m_1$ & $n_1$ \\
    Controls & $0$ & $n_2$ & $n_2$ \\
    \hline
    \end{tabular}
\end{center}
If the p-values do not fall below the desired type I error threshold, then the rejection region (for $O_{11}$) must lie beyond $m_1$. 
Under the fixed marginal assumptions of Fisher's exact test, no contingency tables with the given marginals can be rejected at the specified level.
In other words, we have given up all power to achieve proper type I error control.
Therefore, the minimum counts needed for the risk allele count must exceed $m_1$, in order for association tests to have any power.

For rare non-risk alleles, the most extreme observation corresponds to the following table.
\begin{center}
    \begin{tabular}{cccc}
    \hline
    & \multicolumn{2}{c}{Allele type} \\
    \cline{2-3}
    \# Observations & Variant 1 & Variant 2 & Counts by phenotype\\
    \hline
    Cases & $n_1$ & $0$ & $n_1$ \\
    Controls & $n_2-m_2$ & $m_2$ & $n_2$ \\
    \hline
    \end{tabular}
\end{center}
We can similarly determine the minimum number of non-risk allele counts needed to achieve non-zero power at the given type I error target, for a given phenotype marginals $(n_1, n_2)$.

For correctly calibrated tests, an alternative hypothesis with expected variant counts less than the minimum calibration numbers should have power close to zero; asymptotic power approximations do not apply for these alternatives.
We correct the asymptotic approximations laid out in Section \ref{subsec:power-calculation} by setting the predicted statistical power for alternatives in this ``rare-variant zone'' to zero.

% We conduct an extensive simulation study to examine the quality of this finite-sample correction in Section \ref{sec:numerical}.
% We find this simple rule produces accurate corrections, matching up well to the simulated powers of exact tests.
% In general, the correction kicks in only for small sample sizes.

In the web-based application U-PASS, we mark the ``rare-variant zone'' with red dashed lines in the OR-RAF diagram.
We also provide options for users to specify the rare-variant threshold by absolute number of counts, or as a fraction of the total number of subjects.
% We find these two options ad-hoc; the minimum calibration numbers approach provides a more theoretically grounded approximation in power calculations for rare-variants.

