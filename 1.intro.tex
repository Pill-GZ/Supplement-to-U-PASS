% Power analysis is a crucial in the planning, as well as in reviewing, of genome-wide association studies (GWAS).
There has been no shortage of power calculators for genome-wide association studies (GWAS).
Early efforts by \cite{Sham98} enabled power analysis of likelihood ratio tests for associations between marker SNP's and quantitative or qualitative traits; the results were implemented in the GPC calculator \citep{Purcell03}.
Subsequently, \cite{Skol06} studied the performance of two-sample t-tests, and extended the analysis to two-stage designs; the results were implemented in their CaTS calculator, and more recently, the GAS calculator for one-stage designs \citep{Johnson17}.
\citealp{Menashe08} independently implemented the latter calculations in the PGA calculator.
Recent works have studied power of a number of SNP-set based tests targeting rare variants \citep{Wang14, Derkach17}.
An exhaustive review is beyond the scope of this paper.

It is worth noting that power analyses are, inevitably, tied to the underlying models and the statistical procedures used.
In particular, power calculations performed under a certain model-method combination may not be valid if either the model or the method changes.
This is a rule well-observed in studies comparing powers of SNP-set based methods, but perhaps sometimes forgotten when performing single SNP based association tests.
For example, neither CaTS, GAS, nor PGA are explicit about the testing procedures that should be assumed.
In principle, power calculations based on likelihood ratio tests or t-tests cease to hold for studies running, say, logistic regressions or chi-squared tests.

Along with the multitude of models come conflicting definitions. 
A typical example is risk allele frequency (RAF), for which we identify at least three different definitions, detailed in the Supplement. 
While GWAS Catalogs -- such as the one hosted by NHGRI-EBI \citep{MacArthur16} -- require studies to report RAF \emph{in control groups}, all aforementioned power calculators in fact assume the RAF input as the counterpart \emph{in the general population}, a differently defined quantity.

Along with the multitude of methods also come diverging reporting practices.
While some studies report estimated odds ratios (OR), others report, e.g., estimated slope coefficients from logistic regressions, both of which are accepted in the NHGRI-EBI GWAS Catalog.

As a result, it is not only challenging to use the power calculation tools for planning GWAS studies correctly, 
but also difficult to systematically review the statistical validity of findings reported in the literature, since different models and tests must be handled differently, and with care.

Lastly, we point out the overlooked issue of rare variants in earlier power calculators.
While existing tools rely on large sample approximations, they are completely silent to the failures of such approximations in finite samples.
It is not clear if results from such asymptotic analysis are reliable especially when variant counts are low.

In an effort to address these difficulties and deficiencies, we propose in this work a unified framework for power analysis of single SNP based association studies.
The ideas are implemented in a use-friendly software tool, enabling model-free test-independent power analysis, as well as systematic reviews of reported findings of their statistical validity.
